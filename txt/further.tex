\section{Further research}
\label{further}
There are ideas worth exploring related to ALT that emphasise interactivity 
and assessment. The law of exercise emphasises that in order to achieve the 
best learning results, practice and feedback must 
coexist\cite{murphy2011games}.

An integral part to ALT is that the learner experiences high levels of 
success\cite{cotton1990educational, murphy2011games}. Facilitating ALT is in 
principle straight forward, but for the balancing of difficulty of skill. 
Success requires a delicate balance in which tasks are challenging yet 
achievable. Feedback (the manner in which the learner perceives their 
progress) is intrinsically entangled with achieving this balance, and 
assessment is in turn intrinsically entangled with 
feedback\cite{murphy2011games}.

It would be worthwhile further investigating augmenting non-linearity as a 
means to achieve the difficulty balance. There are numerous angles to 
investigate. For instance, modules may be interactively rearranged or 
hot-swapped based on difficulty.

In Section~\ref{repetition}, naïve self-assessment is suggested. More 
sophisticated methods of assessment are worthy of investigation. One novel 
approach to interactively tutoring learners is Ask-Elle, a programming tutor 
for the Haskell programming language which provide students with feedback on 
incomplete programs, and give hints on how to proceed in order to solve a 
programming exercise\cite{jeuring2012ask}. The avatars of our extended system 
help the system to achieve a more human touch\cite{berntsen2015enabling}, and 
are practical candidates for such a tutoring system, which might double up as 
an assessment tool.

In Section~\ref{synchronisation}, instructor-integration is briefly mentioned. 
Presently, module authors are primarily involved pre-learning. Attempts at 
involving them \emph{during} learning in an instructor role would be 
worthwhile.
