\subsection{Chronicling}
\label{chronology}
The canvas system may be aware of the chronology in a composition. After all, 
it is already there indirectly. As a motivation, let's consider an imagined 
popular canvas. Let $m$, $n$, $o$, $p$, and $q$ be modules. Let $m \to n$, $n 
\to o$, $n \to p$, $o \to p$, and $p \to q$ be possible flows, where $\to$ is 
a binary operator signifying that the user is sent from the module given as 
its left-hand side argument (lhs) to the module given as its right-hand side 
argument (rhs). This canvas is shown in Figure~\ref{canvas}.

\begin{figure}[H]
\begin{centering}
\begin{tikzpicture}[node distance = 6 em, auto]
\node(m){m};
\node[right of = m](n){n};
\node[right of = n](i){};
\node[above of = i](o){o};
\node[below of = i](p){p};
\node[right of = i](q){q};

\draw[thick,->](m)--(n);
\draw[thick,->](n)--(o);
\draw[thick,->](n)--(p);
\draw[thick,->](o)--(p);
\draw[thick,->](p)--(q);
\end{tikzpicture}
\caption{An imagined popular canvas}
\label{canvas}
\end{centering}
\end{figure}

Some self-evident properties here include that the successor of $m = n$, and 
the successor of $n = o \vert p$. Module relations are transitive with respect 
to order, so the successor of $n = q$ via the successor of $o = q$, and the 
successor of $p = q$ --- all the way down to the successor of $m = q$. 
Alternatively, by symmetry, we can say that the predecessor of $q = m$.

Ordering is trivial to store in module metadata, and via very simple 
mathematical operations we are afforded a lot of useful insights. Just from 
the above, several concrete features are easily imagined.

\begin{itemize*}
  \item We can suggest that authors of compositions that contain e.g.\ module 
$m$ might be interested in modules $n$, $o$, $p$, and $q$.
   \item Authors with modules $m$ and $q$ on the canvas could be recommended 
   to insert modules $n$ and $o$ in-between them. This is particularly 
   interesting if the system has other useful metadata. E.g.\ if $q$ is 
   generally considered to be very difficult, and $n$ and $o$ is considered to 
   augment $m$ significantly in preparing for $q$, $n$ and $o$ should be 
   strongly recommended to the author of a canvas with $m$ and $q$ in them.
   \item Authors with modules $m$, $n$, $o$, and $p$, on their canvas could 
   have $p$ recommended as a supplement for $o$ --- or as an alternative to 
   $o$. The latter would be extra useful if we have useful time estimates, 
   forthy it may e.g.\ be that $p$ fits the desired composition estimate 
   better than $o$.
   \item Authors that have module $o$ may be recommended $p$ irrespective of 
   the other modules in their composition, e.g.\ in the aforementioned time 
   constraint  scenario.
\end{itemize*}

As touched on above, combining chronology insights with other metadata, we may 
make several observations.

As an example, let $n : T$, $q : U$, and $o, p : V$, where $:$ can be read 
``has-type'', i.e.\ lhs is a module, and rhs is the type of module (e.g.\ news 
article, scientific paper, video, game, or quiz). Then let $t$ be a function 
that takes a module as its input, and returns the module's type metadata as 
its output. If a user has $n$ and $q$, we can recommend $o$ and $p$ as above. 
But $t o = t p$ means that $V$ is potentially interesting. Thus we can safely 
recommend the set of other modules with the same type as $o$ and $p$, i.e. 
$\left\{ {t m = V \vert m \in M} \right\}$, where $M$ is the set of all 
modules.

If we let $p : W$, and arrive at the conclusion that $p$ should be suggested 
from another metric --- we have that the successor of $n = o \vert p$ in the 
imagined popular canvas, so $o$ and $p$ are similar in other ways than $t o$ 
and $t p$ --- we could recommend $\left\{ t m = t o\text{, or }t m = t p \vert 
m \in M \right\}$, i.e.\ the set of all modules with either $V$ or $W$ as 
their type.

These insights hold for other things than types. As an example take $:$ to be 
``has-topic'', and $t$ to be a function from a module to its topic. The system 
could also look at a combination of different metadata to work out heuristics 
for how to suggest modules.

Satisfied with a sufficient motivation for chronology-awareness, we now turn 
to the implementation of it. The core idea of plaimi's tempuhs system seems to 
be tailored to our use case. The cons are that it needs extended 
expressiveness for the relationships of chronology elements (called timespans 
in tempuhs), and that it has not been put to the test for production use when 
several users are considered. The pros are that it is known to deal with a lot 
of data, and that it offers good guarantees of representing our data 
logically, and preserving said logic. Using tempuhs would allow us to distil 
canvases into timespans, which lets us consider the time aspect carefully. It 
is reasonable to think that several new insights are attainable if we go this 
route.
