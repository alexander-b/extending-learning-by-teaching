\subsection{Repetition}
\label{repetition}
The law of recency states that learning degrades over time, and the law of 
exercise says that learning is increased through 
repetition\cite{murphy2011games}. A common solution to this problem is 
discussed in the paper that initially proposed the 
canvas\cite{berntsen2015enabling}, namely spaced repetition (combined with 
testing); i.e.\ studying across several separated sessions in time rather than 
spending the same amount of time in a single session.

This often leads to higher retention, and is one of the most reliable findings 
in human learning research (echoed in hundreds of studies, the first of which 
dating to the 1800s). It has been predictably demonstrated in both children 
and adults, and for both trivial knowledge (simple facts) as well as advanced 
concepts\cite{carpenter2012using}. It has also been shown to be beneficial in 
realistic (applied) contexts\cite{sobel2011spacing, carpenter2012using}.

As we've already discussed\cite{berntsen2015enabling}, spaced repetition with 
testing is a credible learning method, and thus appealing feature to include 
in our canvas. The initial canvas system was nevertheless designed without an 
emphasis on spaced repetition, in order to provide a more general framework. 
Spaced repetition is usually provided for rather specific and well-defined 
knowledge (translate this word to German, solve this equation for x, etc.), 
and arguably makes less sense for a news article leading up to a discussion. 
The canvas is merely a way to glue things together, where ``things'' is an 
ever so broad term. Another issues is that finding the optimal spacing gap is 
notoriously difficult, and inherently contextual (there is no 
``one-size-fits-all'' solution)\cite{carpenter2012using}.

As it stands, centering the entirety of the canvas's design on spaced 
repetition is unlikely. But it is altogether conceivable to augment it with a 
separate system specialising in spaced repetition. As an example, there is 
nothing that precludes the canvas from facilitating a system which focusses on 
spaced repetition.

The modules and compositions thereof would need metadata tailored to the 
spaced repetition model of learning, but this is not in itself a difficult 
task. The amount of work to at the very least be natively 
spaced-repetition-aware is very low, and the benefits may be 
disproportionately high for systems that may want to use the canvas system. 
Consequently, it is likely a good idea to make at least that much effort.

The next level of effort would be to include a way for learners to self-assess 
retention, in a manner similar to what
Anki\footnote{\url{https://ankiweb.net/about}} and similar programs do. There 
must be a way to mark a module as spaced-repetition-aware, which will then let 
the user self-assess its learning effect at the end of it. This is a feature 
which is useful regardless of spaced repetition, as we are now able to say 
something about our users' retention level based on self-assessment. By 
extension we can say something about the retention success of modules. It 
trivially also follows that we may say something about assessment, both from 
the perspective of a user, and of a module. Assessment can be made into an 
interactive and engaging affair through our system's avatar 
feature\cite{berntsen2015enabling}.

The final step is to actually encourage repeating somehow. This is likely out 
of scope for our system for now. There is however, as mentioned, nothing 
precluding an external system from augmenting our system with this.
