Time plays an integral role in the realm of e-learning modules. plaimi 
previously described a canvas for composing e-learning modules. By 
transitivity, time needs to play an integral role in this canvas. This paper 
investigates some of the ways it can play said role. Four angles are 
considered: insights offered by viewing a module composition as a chronicle of 
modules, the importance of length estimation in time allocation, heightening 
retention via spaced repetition, and synchronisation attempts at facilitating 
collaborative learning. The features discovered by this investigation are 
discussed in a principled learning context, which particularly emphasises 
academic learning time. Some concrete suggestions are made; to implement: 
order-awareness, user estimation of module length, 
spaced-repetition-awareness, and post-module self-assessment. Suggestions for 
further research are also given.
