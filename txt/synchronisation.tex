\subsection{Synchronisation}
\label{synchronisation}
The canvas system is part of a project dubbed ``Learning by Teaching'', and 
was originally conceived as a stepping stone towards cultivating learning by 
teaching, which fosters advantages that do not manifest if the learner relies 
exclusively on an external teacher in a situated learning 
environment\cite{cortese2005learning}. Additionally, the canvas was designed 
to offer an intuitive way of graphically composing e-learning 
modules\cite{berntsen2015enabling}.

The stepping stone satiated by the canvas is authoring. As such, the canvas 
can be said to achieve ``learning by authoring'', a process that covers 
gathering of learning material, and organising. As a result, the canvas 
software focusses on authors.

However, the compositions that are made on the canvas are only interesting 
insofar as they are used. We must therefore not neglect the end-users of 
e-learning in favour of the authors. Although we wish to encourage a learning 
effect from authoring modules, there will be some pure end-users that do not 
author anything. When focussing our attention on these end-users, we must 
consider collaborative learning, as discussion is shown to foster the 
development of critical thinking\cite{gokhale1995collaborative}.

In the interest of collaboration, mechanisms for synchronising users are 
desirable. Three suggestions are discussed here;
\begin{itemize*}
  \item realtime (synchronous) collaboration,
  \item wait-for-me collaboration,
  \item and timeslot collaboration.
\end{itemize*}

Exploring realtime collaboration offers several specific features. The 
collaboration may take place on two levels --- using the modules, discussing 
the modules, or both.

First, let's talk about collaboratively using and discussing modules in 
realtime. The immediate idea here is akin to Twitch Plays Pokémon, where over 
a million users for over two weeks voted on what to do at every step of the 
game \begin{CJK}{UTF8}{min}ポケットモンスター 赤\end{CJK} (Poketto Monsut\={a} 
Aka, known as Pokémon Red outside of Japan), in a strictly egalitarian manner, 
whilst discussing the game in a live chat\cite{tpp}. The experiment is a 
significant phenomenon that demonstrates that social groups are able to unite 
in social contexts where obstacles are presented\cite{margeltwitch}. The 
experiment is largely transferable to users of our canvas system, in that 
there are several modules for which it is possible to have a group of people 
using them at the same time with e.g.\ the majority vote deciding how they 
progress. More sophisticated voting mechanisms such as Condorcet may be 
desirable to provide better overruling heuristics. Discussion may be directly 
transfered; i.e.\ a regular realtime chat is provided.

There are several ways of making this idea more sophisticated. Users may need 
to discuss and argue their views as to why e.g.\ one answer in a quiz is 
correct and others are not, in order to achieve a satisfactory outcome (per 
some voting heuristic), lest they be prevented from progressing. Discussion 
may be augmented with features that make it easy to refer to information 
within a composition. As an example, in a composition where the users are on 
module three, a quiz, they may wish to refer to module two, an article, to 
strengthen their argument. In this example the user needs a simple way of 
accessing previous modules, and a way of easily using them in a discussion.

If the reader is concerned that the idea has become \emph{too} sophisticated 
now, fear not; there are equally many ways of distilling it down into simpler 
components. E.g.\ A chat by itself. This modest feature would be a rather 
large extension of the canvas system. Especially as it was argued against in 
the original implementation to avoid abusive 
behaviour\cite{berntsen2015enabling}.

Instead of each participant actively influencing module outcomes, a seat mode 
may be used. There are several conceivable implementations of this. One is 
that there is one (somehow elected) person in control all the time, that needs 
to act on behalf of the group. Another is a hot seat solution in which the 
seat holder changes based on some heuristic.

Modules that are merely articles or videos or other non-interactive learning 
material arguably benefit the least from realtime collaboration --- fast 
readers must wait for slow readers, and that's about it. This is where 
wait-for-me collaboration becomes useful. The general idea is that there are 
several synchronisation points where users must become synchronised. In the 
example above, it would be natural that if module one and two were reading 
material, these may be pursued independently. There is nothing precluding the 
joint existence of wait-for-me and realtime mechanics, so that once the users 
are synchronised, they may use a module --- such as the quiz in module three 
--- in collaborative realtime. Another useful combination is wait-for-me 
synchronisation at certain intervals, after which realtime discussion takes 
place. But wait-for-me mechanics have useful properties when viewed 
independently as well. One concrete example that is easy to imagine useful is 
in a largely situated learning environment where it is desirable that all 
learners possess roughly the same information.

Timeslot collaboration is another useful idea for situated learning 
environments. It is additionally also useful for learners that want to 
collaborate across timezones, or following some self-imposed schedule. A 
timeslot mechanism would entail completing modules in certain timeslots. 
Teachers often set learning material per class per week in school, and gives 
homework based on rather tight timeslots, so this is a familiar concept. 
Again, it may be combined freely with the other two synchronisation methods. 
It may also be nested. E.g.\ a timeslot to do a module composition wherein 
wait-for-me mechanics are used for non-interactive learning material, 
culminating in a real-time quiz and subsequent discussion.

Where naïve wait-for-me moves in the pace of the slowest participant at the 
risk of alienating the quicker participants, naïve timeslot synchronisation 
moves at a set pace and risks leaving the slower participants behind. These 
problems mean that the features may have exactly the opposite of our intended 
effect, maximising ALT, for some subset of users. Wait-for-me synchronisation 
needs to consider a method of progressing if one (or more) participants are 
slowing the group down, whilst timeslots need to consider a way of ensuring 
that participants are actually learning. Realtime in turn risks virtually all 
known problems with online social interaction\ldots

Collaborative learning is most effective with an instructor that facilitates 
learning\cite{gokhale1995collaborative}. It then follows that we should seek 
to foster learning by instruction in addition to learning by authoring. 
Marrying the two (i.e.\ users acting as instructors of material they have 
themselves authored) gets us much closer to learning by teaching proper.

Consequently synchronisation should be extended to encompass instructors as 
well. There are several ways of achieving this. Instructors may provide 
realtime feedback whilst a group is going through a module, or discussing it. 
They may also act as the seat holder, thereby offering a potential solution to 
any social problems.

This entire section has a certain latent conjecture hanging over it: 
Synchronous collaboration is mostly interesting in a (semi) situated learning 
environment. However, this environment needn't be a classroom setting. It just
needs to be facilitated in the system that surrounds the canvas. Study groups 
or a similar mechanism in which learners may organise may be added, including 
potentially a matchmaking system, and a forum for getting in touch with 
potential collaborators. All of which are major undertakings and nigh-complete 
transformations of the original concept --- this isn't to be taken lightly.

Another problem with uncoordinated collaborators is the possibility of 
upsetting the precarious ALT by effectively making each collaborator adapt to 
each other. This could potentially result in every single collaborator 
following a sub-optimal pace, thereby harming the chances of achieving ALT, 
making this a negative feature rather than positive. Well put-together study 
groups alleviate this slightly, but not completely.

We elect not to explore potential implementation problems in detail in this 
section. Just like the core repetition idea discussed in 
Section~\ref{repetition}, the ideas presented here are of such a magnitude as 
to warrant completely new user experience research. However, it is noteworthy 
that nothing discussed in this section is fundamentally difficult from a 
technological perspective. The user-interface and -experience design 
challenges are far greater (though not insurmountable).

Like we concluded with spaced repetition, it is entirely plausible that 
synchronous collaboration is best left to another system which augments the 
canvas. It may also be suitable as a special part of some expanded system, 
wherein the learning material itself is optimised for a collaborative premise. 
Collaborative material which enables critical thinking and discussion are 
likely to be more successful than users attempting to collaborate on material 
not designed with collaboration in mind.

With learning material optimised for collaboration, it is possible to 
approximate real world tasks to a higher degree. As an example, consider the 
software engineering composition visualised in Figure~\ref{collabcanvas}. Let 
$A$ and $B$ be the participants of this canvas. Let $a$, $b$, $c$, and $d$, be 
modules. Let $a \to b$ and $b \to d$ be flows unique to $A$, and $a \to c$ and 
$c \to d$ unique to $B$. The topic might be compilers. $a$ might be an 
introduction to compilers, then $b$ can be an introduction to frontend 
(lexing, parsing, etc.), and $c$ an introduction to backend (assembling, 
code-generation, etc.), and finally $d$ can be a quiz about both front- and 
backend fundamentals. This models real world collaboration in a sense. It is 
not unusual to divide up tasks like this in software engineering. The quiz 
might now be a realtime collaborative task in which the $A$ and $B$ must rely 
on each others' knowledge in order to pass it. Through this process, it is 
plausible that $A$ will learn about compiler backends, and that conversively 
$B$ will learn about compiler frontends.

\begin{figure}[H]
\begin{centering}
\begin{tikzpicture}[node distance = 6 em, auto]
\node(a){a};
\node[right of = a](i){};
\node[above of = i](b){b};
\node[below of = i](c){c};
\node[right of = i](d){d};

\draw[cyan,thick,->](a) -- (b);
\draw[pink,thick,->](a) -- (c);
\draw[cyan,thick,->](b) -- (d);
\draw[pink,thick,->](c) -- (d);
\end{tikzpicture}
\caption{A canvas optimised for collaboration}
\label{collabcanvas}
\end{centering}
\end{figure}

The interactive nature of our canvas makes collaborative learning a natural 
fit. But the original design did not consider collaborative learning, and as 
such the augmentation must be considered too great to be done recklessly. User 
experience research is thus thoroughly recommended, and indeed necessary.
