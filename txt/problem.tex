\section{Motivation}
When designing a learning experience, it is essential to consider time. This 
includes an awareness of allocated, engaged, and academic learning time (ALT), 
lest dead time incurs for want of understanding. Allocated time is the amount 
of time allocated for learning. Engaged time is time spent actively attempting 
to learn. ALT is time spent engaged in appropriate learning that leads to high 
levels of success\cite{cotton1990educational}; i.e.\ time spent in the flow. 
Flow is the state of being fully immersed and focused on an 
activity\cite{murphy2011games}.

There is a very slight but persistent correlation between allocated time and 
achievement\cite{cotton1981time, walberg1988synthesis, cotton1990educational}. 
Engaged time is modestly correlated with achievement\cite{cotton1981time, 
sanford1983time, cotton1990educational}. ALT leads to more 
learning\cite{walberg1988synthesis}, and the rate of ALT is \emph{highly} 
correlated with achievement\cite{cotton1981time, sanford1983time, 
walberg1988synthesis, cotton1990educational}. Related to our canvas system we 
also take note that interactive engaged time lead to higher achievement than 
non-interactive engaged time\cite{sanford1983time, cotton1990educational}.

Pre-laptop era research demonstrates unequivocally that school pupils only 
spend roughly half of their in-class time engaged in 
learning\cite{cotton1990educational}. Laptop era research shows that students 
with laptops, compared to those without, spend more time engaged in learning, 
develop better critical thinking skills, and are more self-reliant. 
Additionally, laptop users are significantly higher-achieving than their 
non-laptop-using counterparts\cite{cengiz2005learning}. There is no indication 
that this research should not extend to today's era of laptops coexisting with 
tablets and sophisticated mobile phones in the classroom.

When considering the canvas system, we need to seek not only to maximise 
engaged time, but also to allow e-learning modules authors to think about how 
their module will fit into allocated time. Furthermore, situated learning 
environment professionals, e.g.\ teachers at primary schools, necessarily need 
consider allocated classroom time when choosing which modules to use. 

Moreover, and more importantly, the system must strive for the maximisation of 
ALT by eliminating material that is either too easy or too difficult, or 
otherwise unsuitable to the learner. The system admits the possibility of 
non-linear adaptive compositions of e-learning modules --- and it would not be 
unfaithful to the original concept to attach the utmost importance to this 
goal --- leading to a great opportunity to further elevate the present ALT 
ratio both in and out of classrooms.

In addition to maximising learning, we also seek to maximise motivation 
(willingness to engage) and minimise procrastination (unwillingness to engage; 
the absence of (self-regulated) performance\cite{lee2005relationship}). First, 
we suppose that our canvas system is a gamified system, and thus it is 
inherently comparable to a game in several ways\cite{deterding2011game}. This 
is intended design\cite{berntsen2015enabling}. Furthermore, designing 
e-learning modules is a subset of instructional design, which is fundamentally 
similar to game design\cite{murphy2011games}. Then we accept that ALT in our 
system is a form of flow. This is a reasonable conjecture, because flow state 
works by precisely the same mechanics as ALT: performing at the edge of one's 
competency, guided by feedback\cite{rutledgepositive}. This is altogether the 
point of ALT, as designed and desired to manifest in our gamified system. From 
this follows several insights.

Flow has a number of different desirable properties. It is intrinsically 
linked to motivation and widely accepted as one of the fundamental reasons 
people play games\cite{murphy2011games}, and thus an emphasis on flow might 
cause people to use our canvas system. It follows immediately from the law of 
readiness (and indirectly from the law of effect) that learners learn best 
when motivated\cite{murphy2011games}, and the whole point of our system is for 
our users to learn. Additionally, presence of flow is significantly negatively 
correlated with procrastination, and absence of flow is significantly 
positively correlated with procrastination\cite{lee2005relationship}.

Consequently, we must conclude that ALT --- and by extension time --- as a 
concept is intrinsic to our system.
